% This example An LaTeX document showing how to use the l3proj class to
% write your report. Use pdflatex and bibtex to process the file, creating 
% a PDF file as output (there is no need to use dvips when using pdflatex).

% Modified 

\documentclass{l3proj}
\begin{document}

\title{ESE1 Team Project Dissertation}

\author{Agnes Ola \\
        Ross James Gardiner \\
        Lorenzo Roccato \\
        Duncan Lowther \\
        Nawaf Al Lawati}

\date{6 April 2020}

\maketitle

\begin{abstract}

%% The abstract goes here

\end{abstract}

%% Comment out this line if you do not wish to give consent for your
%% work to be distributed in electronic format.
\educationalconsent

\newpage

\section{Marking sheme - remove before submission}

Reflecting on your practice is the hardest part of writing the dissertation, so you are encouraged to talk to the course coordinators and demonstrators to find out what you could include in this section. A good source of examples of incidents for reflection is often the documentation from your retrospectives, because you used the retrospectives to identify areas of your process that could be changed or done better. You should also, try to relate your experiences to other studies available in the software engineering literature (the recommended reading is a good starting point for this).

For example, if you found that you had to drop a feature during an iteration, discuss the reasons why the feature had to be dropped. Had you given yourselves too much work? Was the feature harder to implement than you realised? Had you got your priorities wrong? Then consider looking at the literature (see the recommended reading for PSD3) on project planning and estimation. Was your experience typical of a software project? What steps do other developers advocate for improving estimation?

Alternatively, did you have to make some big design decisions or choice of software platforms early on in the project? What impact did these choices have? Were they the right ones? How might you have improved the decision making process to reduce uncertainty? Did you implement a prototype before proceeding to far with the main implementation? How much effort did this involve? What did you learn about the platform as a result?

The dissertation should be a single PDF document of a maximum of 12 pages, not including front matter and references. You must use the LaTeX template provided for the dissertation and include all the requested meta-data. The LaTeX source for the template, your dissertation and any associated figures should be stored in your version control repository in a clearly indicated directory or branch. It must be possible to build the dissertation PDF from this source, using an automated build script, such as the ant build script provided.

\begin{itemize}
\item Presentation/structure -  	Is the dissertation complete, well organised, clear and literate? Are there examples of spelling mistakes or poor grammar in the dissertation? Is there a clear logical argument and structure to the narrative? Are references used effectively? Are the references complete?
\item Description of objectives and achievements. - Are the problem domain, scope and objectives of the project clear? Is it clear what was achieved during the project and what the key technical challenges were?
\item Reflection -  	Are there a number of experiences/critical events discussed in the dissertation? Does each experience answer the following questions: What was the circumstances of the event? What was learned as a result of the experience? What changed in the project team (if anything) as a result of the experience? Is the experience related to other case studies available in the software engineering literature?
\end{itemize}

%==============================================================================
\section{Introduction}



Software engineering 

This paper presents a case study of... 
%% An introduction, explaining the purpose of the document, a very brief outline of the project and a summary of the structure of the rest of the document (approximately 1 pages).

%% Final paragraph.
The rest of the case study is structured as follows.  Section
\ref{sec:background} presents the background of the case study
discussed, describing the customer and project context, aims and
objectives and project state at the time of writing.  Sections
\ref{sec:Issue1} through Section \ref{sec:Issue3} discuss issues that
arose during the project...

%==============================================================================
\section{Case Study Background}

Include details of 
%% A description of the case study background and context. This should include a description of the project customer (what was the nature of the organisation you were working for), their objectives for the project, and a summary of what was actually achieved. Where appropriate, this section should also make reference to similar related projects in order to make the context clear (approximately 1-3 pages).

\subsection{Our client}
%% The customer organisation and background.
\subsection{Our project}
%%  The rationale and initial objectives for the project.
\subsection{Our software}
%% The final software was delivered for the customer.

%==============================================================================
\section{Reflections}
\subsection{Windowbuilder themed part, not sure about name yet}

Incidents and events:
- Moving things about in the GUI disaasociated comments from corresponding code
- Made merge impossible
- Propagation of early, "temporary" design decisions
- Difficult to rename variables
- Look into why diff was so confused
- Mention abandoned branch merges
- Could only complete changes on the last day afer everyone else was done
- Perhaps something on deadlocks waiting for others to finish

What we learned:
- Make more files
- Balance workload
- Push earlier releases to see all the cracks in the seams
- Acknowledge that there is no such thing as temporary code.

%==============================================================================
\subsection{Agile Testing For Hardware Dependant Software}
 %\\ - Kasper writes about Agile methods for embedded system development. He describes development for hardware dependant systems as challenging because hardware itself cannot be tested in an Agile fashion.
Hirvikoski considers Agile methods for embedded system software development \cite{hirvikoski}. One point highlighted by this report is difficulty in establishing whether a bug is the result of faulty hardware or software. Additionally, Hirvikoski describes hardware development life cycles as generally being much longer than software, often elapsing years between a singe iteration. The literature shows testing is one of the most challenging elements of a hardware dependant Agile software project. NextSteps is no exception. 
%\\ - Agile development styles are designed for rapid deployment.
\\One important characteristic of an Agile code-base is it's frequency to change many times even within a single day.
%\\ - To verify this, a rapid style of testing is required. 
 For this development style, an equally rapid software testing procedure is required to verify changes to the code-base as it mutates. 
%\\ - Rapid testing is vital because it ensures application integrity and developer confidence in the codebase. 
 Rapid testing is a vital form of validation as it not only detects bugs in software quickly and effectively but also ensures developers continuous confidence in the ever changing code-base.
 %\\ - Continuous integration techniques address this need this with the use of pipelines which can verify builds (reference fowler)
\\ For continuous integration, the need for a rapid testing facility is addressed with the use of pipelines which verify builds as they are added to the codebase\cite{fowler1}. 
%\\ - Fowler suggests that testing be done in a clone of the production environment. 
Fowler suggests software pipeline testing should be performed in a clone of the software production environment\cite{fowler2}. For our product, the production environment includes the hardware itself and the world it interacts with.
%\\ - With hardware dependant software this requirement is particularly challenging as building a virtual production environment is often a larger undertaking than the project work itself. 
\\This presents a problem for hardware dependant software such as NextSteps. Given that we are unsure of the technical details of the hardware and exactly how it physically interacts with the world, building a virtual production environment was thought to be a more substantial undertaking than the project work itself. 
%\\ - Here's the issues that we had when testing - specify our own circumstances.
%\\These are similar issues to what the team encountered when configuring a test facility for NextSteps. 
%\\ - Other logic classes rely on a connection to the hardware, which is also not accessible from the pipeline. 
\\Our logic classes rely on a connection to hardware that is not accessible from the CI runner - which operates from a remote server. 
%\\ - CI runner exists in a headless environment. Therefore it is not possible to instantiate our main application code.
Similarly, the CI runner executes in a headless environment, so instantiating the main GUI application was also not possible. 
%\\ - Here's what we did to test the application regardless.
%\\ - Check if the codebase builds. This basic test may be used to highlight compile-time errors
%\\ - Instansiate AAARunner.java, check for its instantiation
As a result of this, our testing pipeline could only compile the code and instantiate a minor class, \verb|AAARunner.java|. 
\\Our approach to testing may have been lacking in the rigour that is nominally required for rapid deployment, but we learned that our technique did have some merit. For example, compile-time errors pushed to the repository were always caught by the CI runner attempting to build the project. 
%\\ - Additionally, the target version of Java is installed on the CI runner, this flags accidental use of outdated java versions which are not compatible with the codebase.  
We also configured the CI runner with the target development version of Java. On a few occasions, errors as a result of prior Java versions on developer PCs were also highlighted by our solution. 
 %\\ - Our testing technique left much to be desired
\\However, this approach still left much to be desired. In fact, on the run up to the final product release, we noticed that the results produced by our application did not match that of the legacy program. Without the subsequent urgent patching, this would have been considered a critical software failure. Winter describes this as "testing crunch-time"\cite{winter}. 
%\\ - Here's what we couldve done better 
%\\ - emulate the hardware.
\\To do better, the team could've made an effort to emulate the hardware as a "Black Box" perhaps emulating some standard data from previous surface tests. This wouldn't account for the response of our product to unexpected data from the hardware, but would at least allow us to emulate the device at some level. Another approach could be to change the pipeline hosting machine to one that isn't headless. This could allow the GUI to be instantiated and perhaps tested using the Java \verb|java.awt.Robot| library to simulate user interaction. However, many online resources advise against this\cite{ruiz}.
%\\ - Run pipeline from machine which is not headless, perhaps test using robot or equivalent. Problem - windows requirement
Another option could be to locate the pipeline server on a machine connected to the hardware. This would be the most accurate way to include the hardware in the synthesised production environment. Finally, in hindsight, the team has learned that static code analysis has many benefits for CI testing and would've been completely feasible within our circumstances. Using this approach allows code to be checked for style compliance, static code metric evaluation and even the detection of some additional bugs\cite{balachandran}. This is something we will take with us into future projects.
%\\ - Run pipeline from machine connected to the hardware.
%\\ - Static code analysis testing
%\\ - How does this relate to the case study? 


%cidents and events
%- Set up yaml / autotesting - it mostly failed due to forgetting to add supporitng files to repo. Medium useful, but mostly a dry run learning exercise
%- Old software was kind of shit at working without the hardware and it was hard to see if the results were the same
%- Very hard to make progress away from St Alwin building because you don't know if you're going in the right direction - therefore progress somewhat sporadic
%- Even when things do work it's difficult to prove (like with the testing exercise)

%What we learned:
%- A hardware emulator would have helped, but would the effort of building one be worth it?
%- Agree on more rigorous manual tests?
%- Agree each person goes in each weekday and tests everyone's progress? Would have helped with familiarity of each other's code too.

%==============================================================================
\subsection{Issue3}

%% Several sections that reflect on your experiences during the team project. Each section should discuss one theme, characterised by incidents or events that occurred during the team course of the project from which you learned (approximately 8-10 pages).
%==============================================================================

\section{Conclusions}
%% A conclusion that draws general and wider lessons from the case study (approximately 1-2 pages).
Explain the wider lessons that you learned about software engineering,
based on the specific issues discussed in previous sections.  Reflect
on the extent to which these lessons could be generalised to other
types of software project.  Relate the wider lessons to others
reported in case studies in the software engineering literature.

%============================================================================== 
\bibliographystyle{plain}
\bibliography{dissertation}
\end{document}
